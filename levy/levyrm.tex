\documentclass[]{article}
\usepackage{amssymb,amsmath}
\usepackage[utf8]{inputenc}
\usepackage[bitstream-charter]{mathdesign}
\usepackage[T1]{fontenc}
\usepackage[dvipsnames]{xcolor}
\usepackage{hyperref}
\hypersetup{
  pdftitle={Notes on L\'evy processes},
  pdfauthor={Kexing Ying},
  colorlinks=true,
  linkcolor=Maroon,
  filecolor=Maroon,
  citecolor=Maroon,
  urlcolor=Maroon,
  pdfcreator={LaTeX via pandoc}}
\usepackage{xcolor}
\usepackage[margin = 1.5in]{geometry}
\usepackage{graphicx}
\usepackage{physics}
\usepackage{amsthm}
\usepackage{mathtools}

\setlength{\parindent}{0pt}
\theoremstyle{definition}
\newtheorem{theorem}{Theorem}
\newtheorem*{theorem*}{Theorem}
\newtheorem*{definition*}{Definition}
\newtheorem{prop}{Proposition}
\newtheorem{corollary}{Corollary}[theorem]
\newtheorem*{remark}{Remark}
\theoremstyle{definition}
\newtheorem{definition}{Definition}[section]
\newtheorem{lemma}{Lemma}[section]
\newtheorem{proposition}{Proposition}[section]
\newtheorem*{proposition*}{Proposition}
\newtheorem{example}{Example}[section]

\setlength{\parskip}{3pt}

\title{Notes on L\'evy processes\\
  \large Based on: \href{link.springer.com/content/pdf/10.1007/978-1-4612-2054-1}{link.springer.com/content/pdf/10.1007/978-1-4612-2054-1}}
\author{Kexing Ying}

\begin{document}

\maketitle

An \(\mathbb{R}^d\)-valued stochastic process \((Z_t)_{t \in [0, T]}\) is said to be a L\'evy process 
if 
\begin{enumerate}
  \item it has independent increments, i.e. for all \(0 = t_0 < t_1 < t_2 < \cdots < t_n \leq T\), the 
    random variables \((Z_{t_{i + 1}} - Z_{t_i})_{i = 0}^{n - 1}\) are independent;
  \item it has stationary increments, i.e. for all \(0 \leq s < t \leq T\) and \(h > 0\), we have 
    \[Z_{t + h} - Z_{s + h} \sim Z_t - Z_h;\]
  \item it is continuous in probability, i.e. for all \(\epsilon > 0\) and \(t \in [0, T]\), we have 
    \[\lim_{h \to 0} \mathbb{P}(\abs{Z_{t + h} - Z_t} > \epsilon) = 0.\]
\end{enumerate}

It is easy to see that, if \((Z_t)\) is an \(L^1\)-L\'evy process is such that \(\mathbb{E}[Z_t] = mt\), 
then \(M_t := Z_t - mt\) is a martingale with respect to its natural filtration. Moreover, if \((Z_t)\) is  
an \(L^2\)-L\'evy process such that \(\mathbb{E}[Z_T^{\otimes 2}] = Vt = (v_{ij}) t\), then \((M_t)\) is a 
square-integrable martingale such that \(\langle M^i, M^j\rangle_t = v_{ij}t\). 

Generally, it turns out that any L\'evy process is a semimartingale and all continuous L\'evy processes 
are Gaussian processes. Thus, combining with the L\'evy characterisation of the Brownian motion, we have 
that a continuous \(L^2\)-L\'evy process is automatically a Brownian motion if \(v_{ij} = \delta_{ij}\).
In contrast to the typical definition of Brownian motion, we note that in Kunita, any continuous 
L\'evy process is a called a Brownian motion.

We take a long detour here to discuss the theory of the Poisson random measure.

Let \((\mathcal{Z}, \mathcal{B})\) be a measurable space and \(\mu\) a \(\sigma\)-finite measure on \(\mathcal{Z}\).
From AP, we defined a Poisson random measure \(N\) with intensity \(\mu\) as a random variable taking values in the space of 
counting measures on \(\mathcal{Z}\) such that for all disjoint \((A_i)_{i = 1}^\infty \subseteq \mathcal{B}\), 
we have that \(N(A_i)\) are \textit{independent} Poisson random variables with intensity \(\mu(A_i)\). 
Recall that this means that, for all \(k \in \mathbb{N}\),
\[\mathbb{P}(N(A_i) = k) = e^{-\mu(A_i)} \frac{(\mu(A_i))^k}{k!}.\]
We observe that this definition automatically requires that
\[\sum_{i = 1}^\infty N(A_i) = N\left(\bigcup_{i = 1}^\infty A_i\right) \sim 
  \text{Poi}\left(\mu\left(\bigcup_{i = 1}^\infty A_i\right)\right) = 
  \text{Poi}\left(\sum_{i = 1}^\infty \mu(A_i)\right)\]
which is the additive property of the Poisson law.

Kunita's definition for the Poisson random measure is a more explicit as he defines it as the induced random 
measure of a Poisson point process. 
\begin{itemize}
  \item Let \(\mathbb{D}_p \subseteq (0, T]\) be a countable set and \(p : \mathbb{D}_p \to \mathcal{Z}\).
    Define the measure \(N_p\) on \([0, T] \times \mathcal{Z}\) such that  
    \[N_p((s, t] \times U) = \abs{\{r \in (s, t] \cap \mathbb{D}_p : p(r) \in U\}}\]
    for all \(s < t \in [0, T]\) and \(U \in \mathcal{B}\).
  \item A Poisson point process \(p\) is a random variable taking values in the space of functions 
    \(\mathbb{D}_p \to \mathcal{Z}\) such that for any \(E_1, \dots, E_n \in \mathcal{B}\) such that 
    for almost every \(\omega\), \(N_{p(\omega)}((0, T] \times E_i) < \infty\) for all \(i = 1, \dots, n\), 
    defining 
    \[X_t = (N_p((0, t], E_i))_{i = 1}^n,\]
    \((X_t)_{t \in [0, T]}\) is a \(n\)-dimensional L\'evy process (note that it is automatically a pure jump process).
  \item The random measure induced by a Poisson point process \(p\): \(\omega \mapsto N_{p(\omega)}\) 
    is called a Poisson random measure.
\end{itemize}

One can show that a Poisson random measure in the Kunita sense is a Poisson random measure in the AP sense 
with intensity \(\mu(A) = \mathbb{E}[N_p((0, 1] \times A)] =: \mathbb{E}[N_1(A)]\). We sketch the proof 
for \(N_t(A) \sim \text{Poi}(t\mu(A))\):

It suffices to show that the characteristic function of \(N_t(A)\) is \(\phi_t(\alpha) = \exp((e^{i\alpha} - 1)t\mu(A))\).
Since \(N_t(A)\) is a L\'evy process, in particular, it is stationary and infinitely divisible 
(\(N_t(A) = \sum_{k = 0}^{n - 1} (N_{\frac{k + 1}{n}t}(A) - N_{\frac{k}{n}t}(A))\)), 
we have by the L\'evy-Khintchine formula that \(\phi_t(\alpha) = e^{t\psi(\alpha)}\) for some function \(\psi\). Thus, defining 
\[M_t^\alpha = \exp(i\alpha N_t(A) - t \psi(\alpha)),\]
we observe \(M_t^\alpha (M_s^\alpha)^{-1} \perp \mathcal{F}_s\) for all \(s < t\) and so, \(M_t^\alpha\) is a martingale.
Hence, defining 
\[Y_t^\alpha = \int_0^t \frac{1}{M_{s-}^\alpha} \dd M_s^\alpha,\]
we have by the It\^o formula applied to \(\log(M_t^\alpha)\) that
\[Y_t^\alpha = (e^{i\alpha} - 1)N_t(A) - t\psi(\alpha)\]
(this step requires us to observe that 
\(\sum_{s \le t} (e^{i\alpha \Delta X_s} - 1) = (e^{i\alpha} - 1)N_t(A)\)). 
Consequently, taking expectations, we have that 
\(\mu(A) = \mathbb{E}[N_1(A)] = -\frac{\psi(\pi)}{2}\) which implies that 
\[\mathbb{E}[N_t(A)] = -\frac{\psi(\pi)}{2} = t \mu(A),\]
and moreover, 
\[\psi(\alpha) = (e^{i\alpha} - 1)\mu(A).\]
Thus, \(\phi_t(\alpha) = \exp((e^{i\alpha} - 1)t\mu(A))\) as required.

Taking \(\mu(A) = \mathbb{E}[N_1(A)]\) as above, we define 
\[\hat N(\dd t \dd z) = \dd t \mu(\dd z) \text{ and } 
  \tilde N(\dd t \dd z) = N(\dd t \dd z) - \dd t \mu(\dd z) = N(\dd t \dd z) - \hat N(\dd t \dd z).\]
We call \(\tilde N\) the compensated Poisson random measure and \(\hat N\) the compensator of 
the Poisson random measure of \(N\).

Similarly defining \(\tilde N_t(A) = \tilde N((0, t] \times A)\), since \(\mathbb{E}[N_t(A)] = t\mu(A)\),
we have by the previous remarks that \(\tilde N_t(A)\) is a square integrable martingale. Moreover, 
\(\langle \tilde N(A)\rangle_t = t \mu(A)\). 

We would now like to integrate with respect to the Poisson random measure. To this end, we define 
integration with respect to the compensated Poisson random measure \(\tilde N\) while integration 
with respect to the compensator \(\hat N\) is classical as it is a deterministic measure.

We make the following observation: Let \(t \in (0, T]\) and \(A, B \in \mathcal{B}\). Then, as 
disjoint sets evaluated under \(\tilde N_t\) are independent, we have that 
\begin{align*}
  \mathbb{E}[\tilde N_t(A) \tilde N_t(B)] 
  & = \mathbb{E}[(\tilde N_t(A \cap B))^2] 
    + \overbrace{\mathbb{E}[\tilde N_t(A \setminus B)]\mathbb{E}[\tilde N_t(B)]}^{= 0}
    + \overbrace{\mathbb{E}[\tilde N_t(A \cap B)]\mathbb{E}[\tilde N_t(B \setminus A)]}^{= 0}\\
  & = \mathbb{E}[(N_t(A \cap B) - t\mu(A \cap B))^2]\\
  & = t\mu(A \cap B)
\end{align*}
where the last equality follows as \(N_t(A \cap B) \sim \text{Poi}(t\mu(A \cap B))\) and so, 
\(\mathbb{E}[(N_t(A \cap B))^2] = t\mu(A \cap B) + t^2 \mu(A \cap B)^2\). Consequently, as 
\(\tilde N_{s, t}\) is independent of \(\mathcal{F}_s\)
\[\mathbb{E}[\tilde N_{s, t}(A) \tilde N_{s, t}(B) \mid \mathcal{F}_s] = 
  \mathbb{E}[\tilde N_t(A) \tilde N_t(B)] - \mathbb{E}[\tilde N_s(A) \tilde N_s(B)]
  = (t - s)\mu(A \cap B).\]
Thus, we have that 
\[\langle \tilde N(A), \tilde N(B)\rangle_t = t \mu(A \cap B)\]
and we have the It\^o isometry for the compensated Poisson random measure:
\[\mathbb{E}\left[\left(\int \psi(z) \tilde N_{s, t}(\dd z)\right)^2\right] 
  = (t - s) \int \psi(z)^2 \mu(\dd z)\]
for any \(\mathcal{F}_s\) measurable function \(\psi\) such that \(\int \abs{\psi}^2 \dd\mu < \infty\).

As the notation \(\tilde N(\dd t \dd z)\) suggest, we would also like to integrate with respect to time.
To this end, we first define the stochastic integral with respect to the Poisson random measure for 
simple functions of the form 
\[g(t, z) = \sum_{i = 1}^n \psi_i(z) \mathbb{1}_{(t_i, t_{i + 1}]}(t)\]
where \(\psi_i\) is a random variable measurable with respect to \(\mathcal{F}_{s_i} \times \mathcal{B}\). We define
\[\int g(t, z) \tilde N_{\dd t}(\dd z) = \sum_{i = 1}^n \int \psi(z) \tilde N_{t_i, t_{i + 1}}(\dd z).\]

We have the following It\^o type isometry for the stochastic integral with respect to the compensated 
Poisson random measure:
\[\left\langle \int g(t, z) \tilde N_{\dd t}(\dd z) \right\rangle_t 
  = \int g(t, z)^2 \hat N_{\dd t}(\dd z) = \int g(t, z)^2 \dd t \mu(\dd z)\]
and 
\[\mathbb{E}\left[\abs{\int g(t, z) \tilde N_{\dd t}(\dd z)}^2\right] 
  = \mathbb{E}\left[\int g(t, z)^2 \hat N_{\dd t}{\dd z}\right]
  = \mathbb{E}\left[\int g(t, z)^2 \dd t \mu(\dd z)\right].\]
With this, denoting \(L^2(\hat N)\) for the set of all predictable random functionals 
\(g : [0, T] \times \mathcal{Z} \times \Omega \to \mathbb{R}\), i.e. \(g\) is measurable with respect to
\(\mathcal{P} \times \mathcal{B}\) where \(\mathcal{P}\) is the predictable \(\sigma\)-algebra on \([0, T] \times \Omega\),
which satisfy 
\[\int g(t, z)^2 \hat N_{\dd t}(\dd z) = \int g(t, z)^2 \dd t \mu(\dd z) < \infty,\]
as the set of simple functions is dense in \(L^2(\hat N)\), we have defined the stochastic integral 
with respect to the compensated Poisson random measure for all \(g \in L^2(\hat N)\). We remark that 
the stochastic integral with respect to the compensated Poisson random measure is a martingale in time.

Finally, for any \(g \in L^2(\hat N)\), we define the stochastic integral with respect to the Poisson 
random measure \(N\) as 
\[\int g(t, z) N_{\dd t}(\dd z) = \int g(t, z) \tilde N_{\dd t}(\dd z) + \int g(t, z) \hat N_{\dd t}(\dd z).\]
I think the above construction might work for any martingale random measures. However, as we used the 
independent increments property a couple of times, the argument will be slightly different.

We also have a version of the It\^o formula for this stochastic integral due to Kunita-Watanabe. I state 
the one-dimensional result here:
Let \(g, h \in L^2(\hat N)\) satisfy \(gh = 0\) (so that \(N_t(h)\) and \(\tilde N_t(g)\) do not jump 
simultaneously) and take
\[X_t = X_0 + \int_0^t \int_\mathcal{Z} g(s, z) \tilde N_{\dd s}(\dd z) + 
    \int_0^t \int_\mathcal{Z} h(s, z) N_{\dd s}(\dd z).\]
Then, for all \(F \in C^2\), we have that  
\begin{align*}
  F(X_t) & = F(X_0)\\ 
  & + \int_0^t \int_\mathcal{Z} \left[F(X_{s-} + g(s, z)) - F(X_{s-})\right] N_{\dd s}(\dd z)\\
  & + \int_0^t \int_\mathcal{Z} \left[F(X_{s} + h(s, z)) - F(X_{s-})\right] N_{\dd s}(\dd z)\\
  & - \int_0^t \int_\mathcal{Z} g(s, z) F'(X_{s-}) \hat N_{\dd s}(\dd z)
\end{align*}

Going back to L\'evy processes, we have that for any \(A \in \mathcal{B}\), by definition \((N_t(A))\) 
is a L\'evy process. On the other hand, starting from a L\'evy process \(Z_t\) on \(\mathbb{R}^n\), 
we can define a Poisson random measure \(N\) by 
\[N_{s, t}(A) = \abs{\{r \in (s, t] : \Delta Z_r \in A\}}\]
for any \(A \in \mathcal{B}(\mathbb{R}^n \setminus \{0\})\). We remove \(0\) since \(\Delta Z\) is zero on 
all but countably many points. We observe that, if for some \(\epsilon > 0\), 
\(A \cap B_\epsilon(0) = \varnothing\), then as all by finitely many jumps of \(Z_t\) are in \(B_\epsilon(0)\), 
we have that \(N_{s, t}(A) < \infty\). Consequently, we have that \(\mu(B_\epsilon^c) = \mathbb{E}[N_1(B_\epsilon^c)] < \infty\).

This construction allows us to represent any L\'evy process using its corresponding Poisson random measure. 
In particular, we have the \textit{unique} L\'evy-It\^o decomposition\footnote{Possible typo in Kunita: \(z\) is 
replaced by an \(x\) in Kunita.}:
\[Z_t = \sqrt{a}W_t + bt + \int_0^t \int_{B_1^c} z N_{\dd s}(\dd z) 
  + \int_0^t\int_{B_1 \setminus \{0\}} z \tilde N_{\dd s}(\dd z)\]
where \(W\) is a standard Brownian motion, \(N\) a Poisson random measure on \([0, T] \times \mathbb{R}^n \setminus \{0\}\) 
with intensity \(\dd t\mu(\dd z)\) which is independent of \(W\). Moreover, \(\mu\) satisfies 
\begin{equation}\label{eq:levy-measure}
  \int \frac{|z|^2}{1 + |z|^2} \mu(\dd z) < \infty
\end{equation}
and we call it the L\'evy measure of \(Z\). We see that \(\frac{|z|^2}{1 + |z|^2} \ge \frac{1}{2}(1 \wedge |z|^2)\) 
and so, Equation~\eqref{eq:levy-measure} implies that \(|z|\mathbb{1}_{B_1\setminus\{0\}} \in L^2(\hat N)\) which is 
required to make sense of the integral \(\int_0^t \int_{B_1 \setminus \{0\}} z \tilde N_{\dd s}(\dd z)\).
Thus, as \(\mu\) determines the Poisson random measure \(N\), 
we have that the triplet \((a, b, \mu)\) uniquely determines the L\'evy process \(Z\) and we call it the 
L\'evy triple of \(Z\).

In the case where the norm of the jumps of \(Z_t\) are bounded below by some positive constant, we have that the 
L\'evy measure is finite and the decomposition can be instead written as 
\[Z_t = W_t + b't + \int_0^t \int_{\mathbb{R}^d \setminus \{0\}} z N_{\dd s}(\dd z).\]

The proof of this is uses a similar idea to proving that \(N_t(A) \sim \text{Poi}(t\mu(A))\). However, before 
sketching the proof, we first observe the following identity regarding integration with respect to a 
Poisson random measure generated by the L\'evy process \(Z\). Denoting \(g \in L^2(\hat N)\), we have that
\[\int_{(0, t]} \int_{\mathbb{R} \setminus \{0\}} g(s, z) N_{\dd s}(\dd z) = \sum_{s \le t} g(s, \Delta Z_s).\]
To see this, one makes the heuristic observation that the measure \(N_t(\mathbb{R} \setminus \{0\})\) gains by 1 
precisely if \(Z\) jumps at time \(t\). Thus, we have the informal formula that 
\(N_{\delta t}(U) = \mathbb{1}_U(\Delta Z_t)\) which can be made rigorous by considering simple functions.
With this in mind, denoting \(M_t^\alpha = \exp(i\alpha Z_t - t \psi(\alpha))\) and 
\(Y_t^\alpha = \int_0^t \frac{1}{M_{s-}^\alpha} \dd M_s^\alpha\) as before, we find\footnote{Possible typo in Kunita: \(Z\) 
is replaced by \(Y\) in Kunita.} 
\[\Delta Y_t^\alpha = \frac{1}{M_{t-}^\alpha}\Delta M_t^\alpha = e^{i\alpha \Delta Z_t} - 1.\]
Hence, denoting \(\tilde Y_t^\alpha\) for the pure jump part of \(Y_t^\alpha\) so that 
\(\tilde Y_t^\alpha = \sum_{s \le t} \Delta Y_s^\alpha\) and \(\hat Y_t^\alpha = Y_t^\alpha - \tilde Y_t^\alpha\) 
for the continuous part, we have that\footnote{I don't see how Kunita got \(\tilde N\) instead of \(N\) here.}
\[\tilde Y_t^\alpha = \sum_{s \le t} (e^{i\alpha \Delta Z_t} - 1) 
  = \int_{(0, t]} \int_{\mathbb{R}\setminus\{0\}} (e^{i\alpha z} - 1) N_{\dd s}(\dd z),\]
and thus, as \(t \mapsto \int_{(0, t]} \int_{\mathbb{R}\setminus\{0\}} (e^{i\alpha z} - 1) \hat N_{\dd s}(\dd z)
= t\mathbb{E}[\exp(i\alpha N_1(\mathbb{R} \setminus \{0\})) - 1]\) has finite variation, we have that 
\[\infty > \langle \tilde Y^\alpha \rangle_t 
  = \left\langle \int_{(0, \cdot]} \int_{\mathbb{R}\setminus\{0\}} (e^{i\alpha z} - 1) \tilde N_{\dd s}(\dd z)\right\rangle_t
  = t\int_{\mathbb{R}\setminus\{0\}} \abs{e^{i\alpha z} - 1}^2 \mu(\dd z)\]
which implies that \(\int \frac{|z|^2}{1 + |z|^2} \mu(\dd z) < \infty\). (?)

Now to determine the shape of \(Z\), we observe that \(M\) satisfy the linear SDE 
\[\dd M_t^\alpha = M_{t-}^\alpha \dd Y_t^\alpha.\]
Thus, using the above decomposition, explicitly write out \(M_t^\alpha\) in terms of \(\hat Y_t^\alpha\) 
and \(N, \tilde N\). 

In general, if \((M_t)\) is a \(L^2\) martingale which is adapted to the natural filtration 
\(\mathcal{F}^Z\) generated by a L\'evy process \(Z_t\), then we have the unique representation
\[M_t = h + \int_0^t f(s) \dd W^i_s + \int_0^t \int_{\mathbb{R} \setminus \{0\}} g(s, z) \tilde N_{\dd s}(\dd z)\]
with \(\int_0^T f^2(s) \dd s < \infty\) and \(g \in L^2(\hat N)\). This shows that the space of 
martingales adapted to \(\mathcal{F}^Z\) can be decomposed into a direct sum \(\mathcal{M}_c \oplus \mathcal{M}_d\)
where 
\[\mathcal{M}_c = \left\{\int_0^t f(s) \dd W_s : \int_0^T f^2(s) v \dd s < \infty \right\}\]
with \(v = \mathbb{E}[W^2_1]\) and
\[\mathcal{M}_d = \left\{\int_0^t \int_{\mathbb{R} \setminus\{0\}} g(s, z) \tilde N_{\dd s}(\dd z) : g \in L^2(\hat N)\right\}.\]
We remark that this sum is indeed a direct sum as the quadratic variation of the Brownian motion and 
the Poisson random measure is zero since the latter is a process of finite variation. 

We state a generalized BDG inequality for L\'evy processes: Let \(X\) be a semimartingale of the form 
\[X_t = x + \int_0^t b(r) \dd r + \int_0^t f(r) \dd W_r + \int_0^t \int_\mathcal{Z} g(r, z) \tilde N_{\dd r}(\dd z).\]
Then,
\begin{align*}
  \mathbb{E}\left[\sup_{s \le t} |X_s|^p\right] & \lesssim_p 
    |x|^p + \mathbb{E}\left[\left(\int_0^t |b(r)| \dd r\right)^p\right] 
    + \mathbb{E}\left[\left(\int_0^t |f(r)|^2| \dd r\right)^{p / 2}\right] \\
    & + \mathbb{E}\left[\left(\int_0^t \int_\mathcal{Z} |g(r, z)|^2 \dd s\mu(\dd z)\right)^{p / 2}\right]
      + \mathbb{E}\left[\int_0^t \int_\mathcal{Z} |g(r, z)|^p \dd s \mu(\dd z)\right].
\end{align*}
We remark that this coincides with the traditional BDG inequality when \(b, g = 0\).

Applying the triangle inequality to \(\sup_{s \le t} |X_s|^p\), the inequality involving all but the last term 
follows from the classical BDG inequality so, we will sketch the proof for the inequality only involving 
the last term. Let \(Y_t = \int_0^t \int_\mathcal{Z} g(s, z) \tilde N_{\dd s}(\dd z)\), applying It\^o's formula 
to \(|Y_t|^p\), we have that 
\begin{equation}\label{eq:ito-normp}
  |Y_t|^p = M_t + \int_0^t \int_\mathcal{Z} \left(|Y_{s-} + g|^p - |Y_{s-}| - p|Y_{s-}|^{p - 2} Y_{s-} g\right) \dd s \mu(\dd z)
\end{equation}
where 
\[M_t = \int_0^t \int_\mathcal{Z} \left(|Y_{s-} + g|^p - |Y_{s-}|\right) \tilde N_{\dd s}(\dd z)\]
is a (local) martingale.

We observe that, by applying MVT twice, there exists some \(\theta \in (0, 1)\) such that 
\[|Y_{s-} + g|^p - |Y_{s-}| - p|Y_{s-}|^{p - 2} Y_{s-} g = p(p - 1)|Y_{s-} + \theta g|^{p - 2}g^2.\]
Hence, using the fact that \((a + b)^n \le 2^n(a^n + b^n)\), we have that 
\[|Y_{s-} + g|^p - |Y_{s-}| - p|Y_{s-}|^{p - 2} Y_{s-} g \le c_3 |Y_{s-}|^{p - 2}g^2 + c_4|g|^p.\]
Hence, taking expectations on both sides of Equation~\eqref{eq:ito-normp}, we have that
\[\mathbb{E}[|Y_t|^p] \le c_3\mathbb{E}\left[\int_0^t \int_\mathcal{Z} |Y_{s-}|^{p - 2}g^2 \dd s \mu(\dd z)\right]
 + c_4\mathbb{E}\left[\int_0^t \int_\mathcal{Z} |g|^p \dd s \mu(\dd z)\right].\]
Applying H\"older's and Young's inequality to the first term provides 
\[\mathbb{E}\left[\int_0^t \int_\mathcal{Z} |Y_{s-}|^{p - 2}g^2 \dd s \mu(\dd z)\right]
  \le \left(1 - \frac{2}{p}\right)\mathbb{E}\left[\sup_{s \le t}|Y_{s-}|^p\right]
  + \frac{2}{p}\mathbb{E}\left[\left(\int_0^t \int_\mathcal{Z} g^2 \dd s \mu(\dd z)\right)^{p / 2}\right].\]
Thus, as \(Y\) is a martingale, by applying Doob's inequality while substituting the above inequalities, 
we have 
\begin{align*}
  & \mathbb{E}\left[\sup_{s \le t}|Y_{s-}|^p\right] \\
  & \le q^p\left(c_3'\mathbb{E}\left[\sup_{s \le t}|Y_{s-}|^p\right]
    + c_3'' \mathbb{E}\left[\left(\int_0^t \int_\mathcal{Z} g^2 \dd s \mu(\dd z)\right)^{p / 2}\right]
    + c_4\mathbb{E}\left[\int_0^t \int_\mathcal{Z} |g|^p \dd s \mu(\dd z)\right]\right)
\end{align*}
where we denote \(q\) for the H\"older conjugate of \(p\). At this point, Kunita shrinks \(c'_3\) so 
that \(q^p c_3' < 1\) and moves the corresponding term to the left hand side to obtain the desired inequality.
I'm not sure why we can shrink \(c_3'\) in this way as it seems to be fixed by the choice of \(p\).

The above lemma allows us to conclude the following approximation for semimartingales of the form 
\[X^x_t = a(x) + \int_0^t b(x, s) \dd s + \int_0^t f(x, s) \dd W_s + \int_0^t \int_\mathcal{Z} g(x, s, z) \tilde N_{\dd s}(\dd z),\]
by semimartingales
\[X^{n, x}_t = a^n(x) + \int_0^t b^n(x, s) \dd s + \int_0^t f^n(x, s) \dd W_s 
  + \int_0^t \int_\mathcal{Z} g^n(x, s, z) \tilde N_{\dd s}(\dd z).\]
In particular, if \(a^n\), \(b^n\) and \(f^n\) converges respectively to \(a\), \(b\) and \(f\) uniformly in time and 
with respect to the \(p\)-th Lipschitz norm, and moreover 
\[\lim_{n \to \infty} \sup_{s \le T} 
  \left\|\left(\int_\mathcal{Z} |g^n(\cdot, s, z) - g(\cdot, s, z)|^{p'} \mu(\dd z)\right)^{1/ p'}\right\|_p = 0\]
for \(p' = 2\) and \(p' = p > d\). Then, for any \(R > 0\), we have that 
\[\lim_{n \to \infty} \mathbb{E}\left[\sup_{|x| \le R} \sup_{s \le T} |X_n(x, s) - X(x, s)|^p\right] = 0.\]
The proof of this approximation is a straightforward application of Kolmogorov's continuity criterion 
and the above BDG inequality.

As a result of this approximation, by a standard Banach fixed point argument, Kunita then concludes 
the existence of a unique solution to an SDE of the form 
\[\dd x_t = b(t, x_t)\dd t + f(t, x_t) \dd W_t + \int_\mathcal{Z} g(t, x, z) \tilde N_{\dd t}(\dd z)\]
where we assume linear growth and Lipschitz conditions on \(b, f\), and moreover, \(g\) satisfies 
\[\frac{g(t, x, z)}{1 + |x|} \le K(z), |g(t, x, z) - g(t, y, z)| \le L(z) |x - y|\]
where 
\[\int_\mathcal{Z} (K(z)^p + L(z)^p) \mu(\dd z) < \infty\]
for all \(p \ge 2\).

Furthermore, under the same conditions, by the BDG inequality, we obtain continuity with respect to the 
initial conditions via Kolmogorov's continuity criterion. 

In the case where our SDE is not just driven by a \(\mathcal{F}^Z\)-martingale (recall the decomposition 
of \(\mathcal{F}^Z\)-martingales into \(\mathcal{M}_c \oplus \mathcal{M}_d\)) but by \(Z\) itself, 
the above result remains to hold. In particular, suppose we consider the SDE with additive L\'evy noise
\begin{equation}\label{eq:levy-sde}
  \dd x_t = b(t, x_t) \dd t + \dd Z_t
\end{equation}
where \(Z\) is a L\'evy process. Then, by the It\^o-L\'evy decomposition, we can write the above 
SDE as
\[\dd x_t = b(t, x_t) \dd t + \int_{B_1 \setminus\{0\}} z \tilde N_{\dd t}(\dd z) + \int_{B_1^c} z N_{\dd t}(\dd z)\]
where \(\int_0^t \int_{B_1^c} z N_{\dd t}(\dd z) = \sum_{0 \le s \le t} \Delta Z_s \mathbb{1}_{B_1^c}(\Delta Z_s)\).
Thus, assuming the growth and Lipchitz conditions, if \(\phi\) is the solution flow of the SDE 
\begin{equation}\label{eq:levy-sde'}
  \dd x_t = b(t, x_t) \dd t + \int_{B_1 \setminus\{0\}} z \tilde N_{\dd t}(\dd z)
\end{equation}
it is possible to construct the solution flow of Equation~\eqref{eq:levy-sde} from \(\phi\). In particular, 
define the random times \((\sigma_n)_{n = 1}^\infty\) where \(\sigma_k\) denotes the \(k\)-th jump time 
for \(Z\) where \(|Z_{\sigma_k}| \ge 1\). Then, fixing \(\omega \in \Omega\), we recursively define the 
solution flow \(\psi\) so that for \(t \in [0, \sigma_1(\omega))\), \(\psi_t(x) = \phi_t(x)\) and for 
\(t \in (\sigma_k(\omega), \sigma_{k + 1}(\omega)]\), we have 
\[\psi_t(x) = \phi_{\sigma_k(\omega), t}(\psi_k(x) + \Delta Z_{\sigma_k(\omega)}).\]
Thus, inductively, as each term is continuous in \(x\) almost surely, we have that \(\psi\) is continuous 
in \(x\). With this argument in mind, it suffices to only consider SDEs of the form \eqref{eq:levy-sde'}.

\end{document}